\documentclass[a4paper,11pt]{article}

\usepackage[utf8]{inputenc}
\usepackage[czech]{babel}
\usepackage[left=2cm,top=3cm,text={17cm,24cm}]{geometry}
\usepackage{graphicx}
\usepackage{listings}
\usepackage{url}

\title{Měření ztrátovosti a RTT\\
{\bf\large ISA - Programování síťové služby}}

\author{Vysoké učení technické v Brně}

\date{\url{https://github.com/ldrahnik/isa_1_project/}}

\date{Lukáš Drahník (xdrahn00), 19.11.2017}

\begin{document}

{\let\newpage\relax\maketitle}

\newpage

\section*{Obsah}
\begin{itemize}
  \item Úvod
  \item Návrh programu
  \item Implementace
  \begin{itemize}
     \item ICMPv4
     \item ICMPv6
     \item UDPv4
     \item UDPv6
     \item UDPv6/v4 server
   \end{itemize}
  \item Rozšíření
  \begin{itemize}
     \item RPM
     \begin{itemize}
        \item Vytvoření .spec souboru
        \item Vytvoření RPM, instalace a odinstalace
     \end{itemize}
     \item Možnost zadání specifického RTT pro uzel
     \begin{itemize}
        \item Upřesnění zadání
        \item Implementace
     \end{itemize}
  \end{itemize}
  \item Výpočet RTT
  \item Příklad použití
  \item Hiearchie projektu
  \item Metriky kódu
  \item Závěr
\end{itemize}

\newpage

\section*{Úvod}
\begin{itemize}
  \item Dokumentace má za účel vysvětlit jak došlo k návrhu programu co se především týče k zadání, jeho implementace a způsob použití. Dále je také věnovaná část k vysvětlení implementace obou z navrhovaných rozšíření.
\end{itemize}

\section*{Návrh programu}
\begin{itemize}
  \item Ze zadání program musí monitorovat jednotlivé uzly paralelně a tedy musí být schopen paralelně odesílat i přijímat přicházející pakety, v případě spínače -l přijímat a odesílat, proto jsou použita vlákna.
  Pro každý uzel je vytvořeno vlákno na odesílání i na přijímání. Jelikož je program tvořen na dlouhodobý běh jsou vlákna v nekonečných smyčkách až do jeho ukončení. Po ukončení vláken dojde k uvolnění alokované paměti.
  Program se nachází v jednom souboru s názvem testovac.cpp a je strukturován do logicky oddělených bloků tvořenými funkcemi.
  \item Z důvodu použitých socketů je potřeba projekt pouštet se sudo.
\end{itemize}

\section*{Návrh programu}
\begin{itemize}
  \item Program pracuje s následujícími inet hlavičkami, které poskytují potřebné struktury, konstanty atd. pro snadnější implementaci:

  \lstset{language=C++}
  \begin{lstlisting}[frame=single,breaklines]

  #include <arpa/inet.h>
  #include <netinet/in.h>
  #include <netinet/ip.h>
  #include <netinet/ip6.h>
  #include <netinet/ip_icmp.h>
  #include <netinet/icmp6.h>

  \end{lstlisting}
\end{itemize}

\section*{Implementace}
\begin{itemize}
  \item Všechny odeslané pakety kromě UDP serveru se ukládají do struktury napříč vlákny z důvodu výpočtu procentuálního podílu ztracených paketů z odeslaných. Náhodná data dána přepínačem -s jsou připojena vždy na konci zásobníku pro odeslání.
  Ve strukturách není všude použito AF\_INET ikdyž by tak být mohlo. AF\_INET je použito pro práci s adresou. Pro práci s socketem, bindování atd. je použito PF\_INET. UDP pakety pracují se soketem SOCK\_DGRAM, ICMP pakety s SOCK\_RAW.
\end{itemize}

\section*{ICMPv4}
\begin{itemize}
  \item Jako hlavička je použitá struktura icmp a icmp\_type zpráv ICMP\_ECHO a ICMP\_REPLY. Timeval pro výpočet RTT se ukládává přímo do struktury do položky. Náhodná data jsou připojena za strukturu icmp.
\end{itemize}

\section*{ICMPv6}
\begin{itemize}
  \item Jako hlavička je použitá struktura icmp6\_hdr a icmp6\_type zpráv ICMP6\_ECHO\_REQUEST a IPPROTO\_ICMPV6. Timeval pro výpočet RTT se ukládává za strukturu icmp6\_hdr. Náhodná data jsou připojena za strukturu timeval.
\end{itemize}

\section*{UDPv4}
\begin{itemize}
  \item Jako hlavička je použitá vlastní struktura outdata\_udp, která obsahuje timeval pro výpočet RTT. Náhodná data jsou připojena za tuto strukturu.
\end{itemize}

\section*{UDPv6}
\begin{itemize}
  \item Kromě úprav pro práci s v6 adresami je implementace stejná jako u UDPv4.
\end{itemize}

\section*{UDPv6/v4 server}
\begin{itemize}
  \item UDP server je naimplementován ve vlastním vlákně, je schopný přijímat jak IPv6 i IPv4 nastavením socket option IPV6\_V6ONLY na false. Přijímá všechny pakety na zadaném portu přepínačem -l
  a ty poté okamžitě odesílá v nezměněné podobě. Adresy akceptuje všechny zadáním in6addr\_any.
\end{itemize}

\section*{Rozšíření}

Byla naimplementována obě z nabízených rozšíření.

\section*{RPM}
\begin{itemize}
  \item K vytvoření RPM balíčku bylo nutné přidat do projektu soubor .spec, který obsahuje mmj. informace o verzi, kde projekt stáhnu (případně by bylo možné stažení zakombinovat do prep sekce),
  jak projekt nainstaluji, s jakými soubory projekt manipuluje. Sekce install odkazuje na soubor Makefile. Aby byly cesty relativní, je souboru Makefile předána proměnná BUILD\_ROOT.

  \item Na instalaci se silně doporučuje použít například alien z důvodu ošetření závislostí atd. Doporučuje se tedy neinstalovat přímo rpm nebo dokonce z Makefile:

  \lstset{language=Bash}
  \begin{lstlisting}[frame=single,breaklines]

  # USING MAKEFILE INSIDE PROJECT (THERE IS NO WAY HOW TO REMOVE, DO NOT DO IT)

  sudo make install


  # DIRECTLY FROM CREATED RPM

  # build rpm
  rpmbuild -ba <package-spec-file>.spec

  # install rpm
  rpm -Uvh <rpm-file>.rpm -v

  # find package
  rpm -qa | grep -i <package-name>

  # uinstall
  rpm -e <package-name>


  # USING ALIEN (HIGHLY RECOMMENDED)

  # build rpm
  rpmbuild -ba <package-spec-file>.spec

  # create .deb
  sudo alien <rpm-file>.rpm

  # install
  sudo dpkg -i <package-name>.deb

  # uinstall
  sudo dpkg --remove <package-name>

\end{lstlisting}
\end{itemize}

\section*{Specifické RTT pro node}
\begin{itemize}
  \item Rozšíření spočívá v možnosti uvedení přepínačem -r pro každý uzel zvlášť a to v syntaxy \textless uzel;RTT \textgreater.
  Pokud dojde k zadání RTT u konkrétního uzlu, je tato hodnota RTT prioritizována nad globální hodnotou pro všechny uzly zadanou přepínačem -r.

  \item Implementace rozšíření spočívala v přidání kontroly zda se jedná o uzel s RTT nebo bez do funkce getNode a případné uložení hodnoty do
  struktury uzlu, která se následně používá dle výše zmíněné priority. K implementaci byly použity strrchr (obdoba funkce strchr) a strtof. První z vyjmenovaných pokud
  najde v řetězci - zprava - výskyt oddělovače, který je v našem případě ; tak vrátí pointer. Dojde tedy k oddělení řetězce na 2 části a o pokus o převedení do float pomocí funkce strtof.
  Je třeba dát pozor na zadávání, v případě středníku v příkazové řádce je potřeba dát celý node do uvozovek nebo znak escapovat následovně:

  \lstset{language=Bash}
  \begin{lstlisting}[frame=single,breaklines]
    sudo ./testovac 'seznam.cz;0.001' google.com\;0.002
  \end{lstlisting}

\end{itemize}

\section*{Výpočet RTT}
\begin{itemize}
  \item Výpočet je nutné provádět pro každé vlákno zvlášť. RTT se počítá minimální, maximální, průměrné a standartní odchylka, která se počítá jako:
\begin{verbatim}
    (SQRT(SUM(RTT*RTT) / N – (SUM(RTT)/N)^2)).
\end{verbatim}
\end{itemize}

\section*{Příklad použití}
\begin{itemize}
  \item Program slouží pro dlouhodobý běh, jakkákoliv průběžně alokovaná paměť během běhu programu (například pro výpis) je okamžitě dealokována. Základní příklad použití při monitorování sebe sama v4 a v6, interval zasílání zpráv 2s, doba vyhodnocování 5s:

  \lstset{language=Bash}
  \begin{lstlisting}[frame=single,breaklines]
    sudo ./testovac 'localhost;0.001' ::1\;0.002 -v -i 200000 -t 5
    2017-11-19 14:29:27.83 104 bytes from localhost (127.0.0.1) time=0.000 ms
    2017-11-19 14:29:27.83 64 bytes from ::1 (::) time=0.000 ms
    2017-11-19 14:29:29.83 104 bytes from localhost (127.0.0.1) time=0.000 ms
    2017-11-19 14:29:29.83 64 bytes from ::1 (::) time=0.000 ms
    2017-11-19 14:29:31.83 104 bytes from localhost (127.0.0.1) time=0.000 ms
    2017-11-19 14:29:31.83 64 bytes from ::1 (::) time=0.000 ms
    2017-11-19 14:29:33.83 104 bytes from localhost (127.0.0.1) time=0.000 ms
    2017-11-19 14:29:33.83 64 bytes from ::1 (::) time=0.000 ms
    2017-11-19 14:29:35.83 104 bytes from localhost (127.0.0.1) time=0.000 ms
    2017-11-19 14:29:35.83 64 bytes from ::1 (::) time=0.000 ms
    2017-11-19 14:29:37.83 64 bytes from ::1 (::) time=0.000 ms
  \end{lstlisting}
  \item

\end{itemize}

\section*{Hiearchie projektu}

/.git \newline
/.vagrant \newline
/doc \newline
/doc/.gitignore \newline
/doc/.local.lib \newline
/doc/Makefile \newline
/doc/manual.aux \newline
/doc/manual.bbl \newline
/doc/manual.blg \newline
/doc/manual.dvi \newline
/doc/manual.log \newline
/doc/manual.pdf \newline
/doc/manual.ps \newline
/doc/manual.tex \newline
/src/testovac.cpp \newline
/src/testovac.o \newline
/.gitignore \newline
/LICENSE \newline
/Makefile \newline
/testovac \newline
/testovac.1 \newline
/testovac.spec \newline
/Vagrantfile \newline

Obsah .gitignore:

\begin{lstlisting}[frame=single,breaklines]
*.o
testovac
*.tar
\end{lstlisting}

Obsah /doc/.gitignore:

\begin{lstlisting}[frame=single,breaklines]
*.aux
*.dvi
*.log
*.blg
*.bbl
*.pdf
*.ps
\end{lstlisting}

\section*{Metriky kódu}
\begin{itemize}
  \item Velikost souboru /src/testovac.cpp: 48557B
  \item Velikost spustitelného souboru /testovac: 37360B
\end{itemize}

\section*{Závěr}
\begin{itemize}
  \item Dokumentace byla vypracována pomocí Latex. Program je překládaný překladačem g++. Program byl otestován lokálně na Ubuntu16.04, po síti na Ubuntu16.04 a v CentOS virtuálním prostředí (předvytvořený Vagrantfile defaultně sdílí složku ze které se spouští,
  proto došlo k přesunutí do projektu), které bylo vyžadováno k běhu programu (po tomto otestování došlo například na odebrání knihovny pro string).
  \item Projekt může sloužit jako analyzátor chování sítě.
\end{itemize}

\nocite{*}

%% BIBLIOGRAPHY
\bibliography{local}
\bibliographystyle{plain}

\newpage
\thispagestyle{empty}

\end{document}
%% END OF FILE
